\documentclass[12pt,-letter paper]{article}
\usepackage{siunitx}
\newcommand{\cosec}{\,\text{cosec}\,}
\usepackage{setspace}
\usepackage{gensymb}
\usepackage{xcolor}
\usepackage{caption}
%\usepackage{subcaption}
\doublespacing
\singlespacing
\usepackage[none]{hyphenat}
\usepackage{amssymb}
\usepackage{relsize}
\usepackage[cmex10]{amsmath}
\usepackage{mathtools}
\usepackage{amsmath}
\usepackage{commath}
\usepackage{amsthm}
\interdisplaylinepenalty=2500
%\savesymbol{iint}
\usepackage{txfonts}
%\restoresymbol{TXF}{iint}
\usepackage{wasysym}
\usepackage{amsthm}
\usepackage{mathrsfs}
\usepackage{txfonts}
\let\vec\mathbf{}
\usepackage{stfloats}
\usepackage{float}
\usepackage{cite}
\usepackage{cases}
\usepackage{subfig}
%\usepackage{xtab}
\usepackage{longtable}
\usepackage{multirow}
%\usepackage{algorithm}
\usepackage{amssymb}
%\usepackage{algpseudocode}
\usepackage{enumitem}
\usepackage{mathtools}
%\usepackage{eenrc}
%\usepackage[framemethod=tikz]{mdframed}
\usepackage{listings}
%\usepackage{listings}
\usepackage[latin1]{inputenc}
%%\usepackage{color}{   
%%\usepackage{lscape}
\usepackage{textcomp}
\usepackage{titling}
\usepackage{hyperref}
%\usepackage{fulbigskip}   
\usepackage{tikz}
\usepackage{graphicx}
\lstset{
  frame=single,
  breaklines=true
}
\let\vec\mathbf{}
\usepackage{enumitem}
\usepackage{graphicx}
\usepackage{siunitx}
\let\vec\mathbf{}
\usepackage{enumitem}
\usepackage{graphicx}
\usepackage{enumitem}
\usepackage{tfrupee}
\usepackage{amsmath}
\usepackage{amssymb}
\usepackage{mwe} % for blindtext and example-image-a in example
\usepackage{wrapfig}
\graphicspath{{figs/}}
\providecommand{\mydet}[1]{\ensuremath{\begin{vmatrix}#1\end{vmatrix}}}
\providecommand{\myvec}[1]{\ensuremath{\begin{bmatrix}#1\end{bmatrix}}}
\providecommand{\cbrak}[1]{\ensuremath{\left\{#1\right\}}}
\providecommand{\brak}[1]{\ensuremath{\left(#1\right)}}
\begin{document}
\begin{center}                                                              \textbf{MATRIX}                                                     \end{center}                                                                                                                         \begin{enumerate}                                                           \item If $A$ is a square matrix of order 2 and \(|A| = -2\), then the value of \(|5A'|\) is:                                                            \begin{enumerate}[label={$\brak{\Alph*}$}]
        \item $-50$
        \item $-10$                                                             \item $10$                                                              \item $50$                                                          \end{enumerate}                                                                                                                                 \item The product of matrix $P$ and $Q$ is equal to a diagonal matrix. If the order of matrix $Q$ is \(3 \times 2\), then the order of matrix $P$ is:                                                                   \begin{enumerate}[label={$\brak{\Alph*}$}]                                  \item $2 \times 2$                                                      \item $3 \times 3$                                                      \item $2 \times 3$                                                      \item $3 \times 2$                                                          \end{enumerate}                                                                                                                        \item If the inverse of the matrix $\myvec{7 & -3 & -3 \\ -1 & 1 & 0 \\ -1 & 0 & 1}$ is the matrix $\myvec{1 & 3 & 3 \\ 1 & \lambda & 3 \\ 1 & 3 & 4}$, then the value of $\lambda$ is:
     \begin{enumerate}[label={$\brak{\Alph*}$}]                               \item $-4$                                                          \item $1$                                                              \item $3$
        \item $4$                                                         \end{enumerate}                                                   
    \item Find the matrix $A^2$, where $A = [a_{ij}]$ is a \(2 \times 2\) matrix whose elements are given by \(a_{ij} = \text{maximum}(i, j) - \text{minimum}(i, j)\):                                                  \begin{enumerate}[label={$\brak{\Alph*}$}]
       \item $\myvec{0 & 0 \\ 0 & 0}$
        \item $\myvec{0 & 1 \\ 1 & 0}$
        \item $\myvec{1 & 0 \\ 0 & 1}$
        \item $\myvec{1 & 1 \\ 1 & 1}$
    \end{enumerate}

    \item If $A$ is a square matrix of order $3$ such that the value of \(|\text{adj}.A| = 8\), then the value of \(|A^T|\) is:
            \begin{enumerate}[label={$\brak{\Alph*}$}]
        \item $\sqrt{2}$
        \item $-\sqrt{2}$
        \item $8$
        \item $2\sqrt{2}$
    \end{enumerate}

    \item If $A$ and $B$ are events such that $P$\brak{A/B}$ = P$\brak{B/A}$ \neq 0$, then:
            \begin{enumerate}[label={$\brak{\Alph*}$}]
        \item $A \subset B, \text{ but } A \neq B$
        \item $A = B$
        \item $A \cap B = \phi$
        \item $P(A) = P(B)$
    \end{enumerate}
\end{enumerate}
\begin{center}
        \textbf{DERIVATIONS}                                            \end{center}                                                  \begin{enumerate}                                                                         \item                                                         [1.]Derivative of $e^{\sin^2x}$ with respect to $ \cos{x}$ is:                                                                                            \begin{enumerate}[label={$\brak{\Alph*}$}]                                                                                          \item $\sin{x}e^{\sin^2x}$                                                          \item $\cos{x}e^{\sin^2x}$                            \item $-2{\cos{x}e^{\sin^2x}}$                                                            \item $-2{\sin^2{x}\cos{x}e^{\sin^2x}}$           \end{enumerate}                                                           \item                                                         [2.]If $\sin $\brak{xy}$=1$,then $\frac{dy}{dx}$ is equal to:                                               \begin{enumerate}[label={$\brak{\Alph*}$}]
                                                          \item $\frac{x
}{y}$                                        \item $-\frac{
x}{y}$
                                        \item $\frac{y}{x}$
                                        \item $-\frac{
y}{x}$                                                                                                  \end{enumerate}
                        \item
                                [3.]The general solution of the differential equation \\ \\ $\frac{dy}{dx}=e^{x+y}$ is:
                        \begin{enumerate}[label={$\brak{\Alph*}$}]
                                        \item $e^x+e^{-y}={c}$
                                        \item ${e^{-x}+e^{-y}}={c}$
                                        \item $e^{x+y}={c}$
                                        \item $2e^{x+y}={c}$                                            \end{enumerate}

        \end{enumerate}
        \begin{center}                                                          \textbf{EQUATIONS}                                                      \end{center}                                                    \begin{enumerate}[label=\Alph*.]                                                \item                                                                           [1.]Given a curve $y=7x-x^3 $ and $x$ increases at the rate of 2 units per second.                                                              The rate  at which the slope of the curve  is changing,when $x=5$ is:                                                           
                \begin{enumerate}[label={$\brak{\Alph*}$}]                              \item$-60$ units/sec                                                            \item$60$ units/sec                                                     \item$-70$ units/sec                                                    \item$-140$ units/sec
                \end{enumerate}                                                 \item
                [2.]The area of the region bounded by the curve $y^2=4x$ and $x=1$ is:                                                                          \begin{enumerate}[label={$\brak{\Alph*}$}]                                      \item $\frac{4}{3}$                                                     \item $\frac{8}{3}$                                                     \item $\frac{64}{3}$                                                    \item $\frac{32}{3}$                                            \end{enumerate}                                                 \item                                                                           [3.]The angle which the line $\frac{x}{1}$=$\frac{y}{-1}$=$\frac{z}{0}$ makes with the positive direction of  Y-axis is:                        \begin{enumerate}[label={$\brak{\Alph*}$}]                                      \item $\frac{5\pi}{6}$                                                  \item $\frac{3\pi}{4}$                                                  \item $\frac{5\pi}{4}$                                                  \item $\frac{7\pi}{4}$                                          \end{enumerate}                                                                                                                 \end{enumerate}                                                                      \begin{center}                                             \textbf{FUNCTIONS}                                                      \end{center}
\begin{enumerate}
        \item
                [1.]The function f$\brak{x}$=$\frac{x}{2}$+$\frac{2}{x}$
has a local minima at $x$ is equal to:

                \begin{enumerate}[label={$\brak{\Alph*}$}]
                        \item$2$
                        \item$1$
                        \item$0$
                        \item$-2$
                \end{enumerate}
        \item
                [2.]A function $f:|R->|R $ defined as f$\brak{x}$=$x^2-4
x+5$ is:
                \begin{enumerate}[label={$\brak{\Alph*}$}]
                        \item injective but not surjective.
                        \item surjective but not injective.
                        \item both injective and surjective.
                        \item neither injective nor surjective.
                                \end{enumerate}                                                                                                 
\end{enumerate}
\begin{center}                                                                      \textbf{INTEGRATIONS}                        \end{center}                                                                                                                                    \begin{enumerate}                                                               \item The value of  \(\int_{\frac{\pi}{4}}^{\frac{\pi}{2}} \cot \theta \, \cosec^2 \theta \, d\theta\) is:                                      \begin{enumerate}[label={$\brak{\Alph*}$}]                                                                              \item $\frac{1}{2}$
                                                        \item $ -\frac{1
}{2} $

                \item $ 0 $                                                                                                                                                     \item $ -\frac{\pi}{8} $                                                                                                                                    \end{enumerate}                                                                                                                                                             \item The integral $ \int \frac{dx}{\sqrt{9-4x^2}} $ is equal to:                                                                                                                                                                               \begin{enumerate}[label={$\brak{\Alph*}$}]
                                                                                                                                                        \item $ \frac{1}{6} \sin^{-1} $\brak{\frac{2x}{3}}$ + c $       

    \item $ \frac{1}{2} \sin^{-1} $\brak{\frac{2x}{3}}$ + c $
                                                                                                                                                
        \item $\sin^{-1} $\brak{ \frac{2x}{3}}$ + c $                           \item $ \frac{3}{2} \sin^{-1} $\brak{ \frac{2x}{3} }$ + c $                                                                             

\end{enumerate}
\end{enumerate}
\begin{center}                                               \textbf{VECTORS}                                                    \end{center}                                                            \begin{enumerate}
        \item
                The Cartesian equation of the line passing through the point $\brak{1,-3,2}$ and parallel to the line:                                           \\ \\                                                                  $\vec{r} = $\brak{2+\lambda}$ \hat{i} + \lambda \hat{j}
+ $\brak{2\lambda-1}$ \hat{k}$ is
      \begin{enumerate}[label={$\brak{\Alph*}$}]                            \item  $ \frac{x-1}{2} = \frac{y+3}{0} = \frac{z-2}{-1}$                \item  $\frac{x+1}{2} = \frac{y-3}{1} = \frac{z+2}{2}$                  \item $ \frac{x+1}{2} = \frac{y-3}{0} = \frac{z+2}{-1}$                 \item  $ \frac{x-1}{1} = \frac{y+3}{1} = \frac{z-2}{2}$             \end{enumerate}                                                             \item                                                                           The position vectors of points $P$ and $ Q $ are \(\vec{p} \) and \( \vec{q} \) respectively. The point \( R \) divides the line segment  PQ  in the ratio 3:1  and $S$ is the mid-point of  line segment$ PR$. The position vector of $ S $ is:                                     
                \begin{enumerate}[label={$\brak{\Alph*}$}]                      \item $\frac{\vec{p} + 3\vec{q}}{4}$                                    \item $\frac{\vec{p} + 3\vec{q}}{8}$
        \item $\frac{5\vec{p} + 3\vec{q}}{4}$
        \item $\frac{5\vec{p} + 3\vec{q}}{8}$
    \end{enumerate}                                                          \item
    \textbf{Assertion $\brak{A}$ :} The vectors \\ \\
                                 $\vec{a}=6\hat{i}+2\hat{j}-8\hat{k}$ \\ \\                                            $\vec{b} = 10\hat{i}-2\hat{j}-6\hat{k}$  \\ \\                                        $\vec{c} = 4\hat{i}-4\hat{j}+2\hat{k}$  \\ \\                                         represent the sides of a right angled triangle.                       \item [] \textbf{Reason $\brak{R}$    :} Three non-zero vectors of which none of two are collinear forms a triangle if their resultant is zero vector or sum of any two vectors is equal to the third.
            \end {enumerate}
\end{document}
